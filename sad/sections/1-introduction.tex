\section{Introduction}

% Checked in grammarly
This document is the Software Specification Requirement (SRS) of a website designed to help earthquake victims to acquire the necessary information and give volunteers a chance to donate to help earthquake victims. The website is called \afetbilgi \cite{afetbilgi}, developed by Middle East Technical University (METU) students and graduates.

\subsection{Purpose and Objectives of \afetbilgi}

% Checked in grammarly
\afetbilgi, direct translation to English is `disaster documentation', is an open-source efforted project led by students from METU in Ankara, Turkiye. It aims to provide a clean, verified, and correctly classified information interface for earthquake victims and helpers alike in the aftermath of the tragic earthquake on February 6th, 2023, in Pazarcik, Turkiye. It also offers quick information using confirmed website links, maps, and address tables, along with the relevant contact details of organizations and helpers involved.

\subsection{Scope}

% Checked in grammarly
\afetbilgi\ was established to offer as much information as needed by users in three main categories:
\begin{itemize}
  \item People who are affected by the earthquake (the victims).
  \item Individuals/Organisations who want to help and participate in other government/private efforted procedures in the affected areas.
  \item People from METU who verify and checked any presented links on the websites.
\end{itemize}

The website is primarily responsible for providing tables and datasheets with website links to third-party organizations/contacts details of web places/physical locations which offer/collect help. As indicated here, these links are external and lead out to other websites(outside from \afetbilgi) whose efforts are verified by human resolves (METU students/helpers/site administrators) on the surface-level user experience.

Given how the world is connected with the internet and phones/televised communication, the project developers aim to create a website using these advantageous characteristics via a simple interface in multiple available languages to create fast and easy use of information with no additional and unnecessary obstacles. In areas lacking internet infrastructure that might have been disturbed by the earthquake activities, the website can be distributed via printed-out PDFs, which are shareable via ordinary computers and mobiles, and hand-forwarded physical versions in the forms of leaflets and so on.

Lastly, \afetbilgi\ includes a map functionality if the victim/helper has an internet connection. Any user can locate helper geolocations via terrain/road routes while also being able to quickly view extra details such as written addresses, contact phone details, and previous reviews.

\subsection{Stakeholders and Their Concerns}

% Checked in grammarly
There are three main categories of people related to \afetbilgi:
\begin{enumerate}
  \item \textbf{Earthquake victims/ affectees:} These individuals whom the earthquake has directly impacted seek help, support, and information to recover from the disaster. They may be looking for information on how to find shelter, food, medical assistance, and other resources that can help them get back on their feet. The website may provide them with a platform to connect with relief organizations and volunteers and access information on navigating the recovery process.
  \item \textbf{Volunteers:} These individuals want to offer their time, skills, and resources to support the relief and recovery efforts. They may include local volunteers, international volunteers, and disaster response teams. The website may attract volunteers by providing information on how to get involved, where to go, and what support is needed. Their primary use of the website could be to scout places to help from outside the main areas, such as centers transporting essential needs to stricken areas like farther cities such as Ankara and Istanbul. This is the target sector for the Donate or Help category, such as via blood donation, monetary donation, physical volunteer help, etc. Other entities such as relief organizations, government agencies, more prominent sponsors, and potential media outlets can exist within this category.
  \item \textbf{Web developers, Data Collectors, and Site administrators:} These are the website creators responsible for developing, designing, and maintaining the platform. They may include web developers, designers, and other professionals involved in creating and managing the website. These stakeholders may be vested in ensuring the website is accessible, user-friendly, effective, and, most importantly, providing simple, verified information to facilitate relief and recovery efforts without any hurdles.
\end{enumerate}
